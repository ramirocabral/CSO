\section{Conceptos Basicos}
\begin{itemize}
    \item \textbf{Sistema Operativo:} Software que actua como intermediario entre el usuario de una computadora y su hardware.
        \end{itemize}
        \subsection{Objetivos de un SO}
        \begin{itemize}
            \item \textbf{Comodidad:} Hacer mas facil el uso del hardware.
            \item \textbf{Eficiencia:} Hacer un uso mas eficiente de los recursos del sistema.
            \item \textbf{Evolucion:} Permitira la introduccion de nuevas funciones al sistema sin interferir con funciones anteriores.
        \end{itemize}
        \subsection{Perspectiva desde el usuario}
        \begin{itemize}
            \item Abstrraccion con respecto a la arquitectura.
            \item El SO "oculta" el hardware y presenta a los programas abstracciones mas simples de manejar.
            \item Los programas de aplicacion son los clientes del SO.
        \end{itemize}
        \subsection{Perspectiva desde la administracion de recursos}
        \begin{itemize}
            \item Administra los recursos de HW de uno o mas procesos.
            \item Probee un conjunto de servicios a los usuarios del sistema.
            \item Maneja la memoria secundaria y los dispositivos de I/O.
            \item Ejecucion simultanea de procesos.
            \item Multiplexacion en tiempo (CPU) y en espacio (memoria).
        \end{itemize}

\subsection{Componentes de un SO}
\begin{itemize}
    \item \textbf{Kernel}
    \item \textbf{Shell}
    \item \textbf{Herramientas}
\end{itemize}

\subsubsection{Kernel}
\begin{itemize}
    \item Porcion de codigo que se encuentra en memoria principal y se encarga de la administracion de los recursos.
    \item Implementa servicios esenciales:
        \begin{itemize}
            \item Manejode la memoria y la entrada/salida.
            \item Manejo de la CPU.
            \item Administracion de procesos.
        \end{itemize}
\end{itemize}

\subsection{Servicios de un SO}
\begin{itemize}
    \item Administracion y planificacion del procesador.
    \item Administracion de la memoria.
    \item Administracion del almacemaniento/sistema de archivos.
    \item Administracion de dispositivos.
    \item Deteccion de errores y respuestas.
        \begin{itemize}
            \item Errores de HW internos y externos.
            \item Errores de SW.
            \item Incapacidad del SO para conceder una solicitud de una aplicacion.
        \end{itemize}
        \item Interaccion con el usuario (Shell).
        \item Telemetria.
\end{itemize}
\subsection{Errores que un SO debe evitar}
\begin{itemize}
    \item Que un proceso se apropie de la CPU.
    \item Que un proceso intente ejecutar instrucciones de E/s por ejemplo.
        \item Que un proceso intente acceder a una direccion de memoria que no le corresponde.
\end{itemize}


\pagebreak
