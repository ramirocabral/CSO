\pagebreak

\section{Conceptos Básicos}
\begin{itemize}
    \item \textbf{Sistema Operativo:} Software que actúa como intermediario entre el usuario de una computadora y su hardware.
\end{itemize}

\subsection{Objetivos de un SO}
\begin{itemize}
    \item \textbf{Comodidad:} Hacer más fácil el uso del hardware.
    \item \textbf{Eficiencia:} Hacer un uso más eficiente de los recursos del sistema.
    \item \textbf{Evolución:} Permitirá la introducción de nuevas funciones al sistema sin interferir con funciones anteriores.
\end{itemize}

\subsection{Perspectiva desde el usuario}
\begin{itemize}
    \item Abstracción con respecto a la arquitectura.
    \item El SO "oculta" el hardware y presenta a los programas abstracciones más simples de manejar.
    \item Los programas de aplicación son los clientes del SO.
\end{itemize}

\subsection{Perspectiva desde la administración de recursos}
\begin{itemize}
    \item Administra los recursos de HW de uno o más procesos.
    \item Provee un conjunto de servicios a los usuarios del sistema.
    \item Maneja la memoria secundaria y los dispositivos de E/S.
    \item Ejecución simultánea de procesos.
    \item Multiplexión en tiempo (CPU) y en espacio (memoria).
\end{itemize}

\subsection{Componentes de un SO}
\begin{itemize}
    \item \textbf{Kernel}
    \item \textbf{Shell}
    \item \textbf{Herramientas}
\end{itemize}

\subsubsection{Kernel}
\begin{itemize}
    \item Porción de código que se encuentra en memoria principal y se encarga de la administración de los recursos.
    \item Implementa servicios esenciales:
        \begin{itemize}
            \item Manejo de la memoria y la entrada/salida.
            \item Manejo de la CPU.
            \item Administración de procesos.
        \end{itemize}
\end{itemize}

\subsection{Servicios de un SO}
\begin{itemize}
    \item Administración y planificación del procesador.
    \item Administración de la memoria.
    \item Administración del almacenamiento/sistema de archivos.
    \item Administración de dispositivos.
    \item Detección de errores y respuestas.
        \begin{itemize}
            \item Errores de HW internos y externos.
            \item Errores de SW.
            \item Incapacidad del SO para conceder una solicitud de una aplicación.
        \end{itemize}
    \item Interacción con el usuario (Shell).
    \item Telemetría.
\end{itemize}

\subsection{Errores que un SO debe evitar}
\begin{itemize}
    \item Que un proceso se apropie de la CPU.
    \item Que un proceso intente ejecutar instrucciones de E/S, por ejemplo.
    \item Que un proceso intente acceder a una dirección de memoria que no le corresponde.
\end{itemize}

\pagebreak
