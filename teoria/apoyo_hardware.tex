\section{Apoyo del Hardware}
\begin{itemize}
    \item Modos de Ejecucion: Limitaciones en el conjunto de instrucciones que se puede ejecutar en cada modo.
    \item Interrupcion de Clock: Se debe evitar que un proceso se apropie de la CPU.
    \item Protecicion de la Memoria: Se deben definir limites de memoria a los que puede acceder cada proceso.
\end{itemize}

\subsection{Modos de Ejecucion}
\begin{itemize}
    \item Un bit en la CPU indica el modo actual.
    \item Las intrucciones privilegiadas solo pueden ejecutarse en modo \textbf{supervisor/Kernel}.
    \item En modo \textbf{Usuario}, el proceso puede acceder solo a su espacio de direcciones, es decir, a las direcicones "propias".
    \item El kernel del SO se ejecuta en modo supervisor.
    \item El resto del SO y los programas de usuario se ejecutan en modo usuario.
\end{itemize}

\subsubsection{Modo Kernel}
\begin{itemize}
    \item \textbf{Gestion de procesos:} Creacion y terminacion, planificacion, intercambio, sincronizacion y soporte para la comunicacion entre procesos.
    \item \textbf{Gestion de memoria:} Reserva de espacio de direcciones para los procesos, Swapping, Gestion de Paginas.
    \item \textbf{Gestion E/S:} Gestion de buffers, reserva de canales de E/S y de dispositivos de los procesos.
    \item \textbf{Funciones de soporte:} Gestion de interrupciones, auditoria, monitoreo.
    \item Cada vez que comienza a ejecutarse un proceso de usuario, el bit de modo se debe poner en modo usuario.
    \item Cuando hay una trap, el bit de modo se pone en modo Kernel. Esta es la unica forma de pasar a modo Kernel.
\end{itemize}
\subsubsection{Modo Usuario}
\begin{itemize}
    \item Debug de procesos, defginicion de protocolos de comunicacion, gestion de aplicaciones.
    \item Tareas que no requieran accesos privilegiados.
    \item No se puede interactuar con el hardware.
    \item Cada proceso trabaja en su propio espacio de direcciones.
\end{itemize}

\subsection{Proteccion de la E/S}
\begin{itemize}
    \item Las instruccines de E/S se definen como privilegiadas.    
    \item Deben ejecutarse en Modo Kernel.
        \item Los procesos de usuario realizan E/S a traves de System Calls.
\end{itemize}

\subsection{Proteccion de la CPU}
\begin{itemize}
    \item Uso de interrupcion por clock para evitar que un proceso se apropie de la CPU.
    \item las instrucciones que modifican el funcionamiento del reloj son privilegiadas.
\end{itemize}



\subsection{System Calls}
\begin{itemize}
    \item Forma en que los programas de usuario acceden a los servicios del SO.
    \item Los parametros asociados a las llamadas pueden pasarse de varias maneras: por registros, bloques, tablas en memoria o la pila.
    \item Se ejecutan en modo kernel.
\end{itemize}

