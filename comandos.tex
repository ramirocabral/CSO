\documentclass[11pt]{article}
\usepackage[obeyspaces]{url}

%! TEX root = "~/home/ramiro/Desktop/CSO/comandos.tex"

\title{\Huge{Conceptos de Sistemas Operativos\\
Comandos utiles en GNU/LInux}}
\author{\huge{Ramiro Cabral}}
\date{\today}

\begin{document}
\maketitle
\pagebreak

\section*{chown}

\textbf{\path{chown user:group /file}}\\

Nos permite cambiar el usuario y/o grupo de un archivo.

\section*{chgrp}
Permite cambiar el grupo dueno de un archivo.

\section*{chmod}
Cambiar los permisos de un archivo (r/w/x).

\section*{df}
Nos informa el espacio usado por los filesystems.

\section*{du}
Informa el espacio en disco usado por cada archivo dentro del directorio especificado.

\section*{who}
Nos muestra los usuarios conectados al sistema.

\section*{su}
El comando nos permite ejecutar comandos con el GID o GDI de otro usuario temporalmente.

\section*{passwd}
Nos permite cambiar la password de un usuario.

\section*{kill}
Finaliza un proceso mediante su PID.

\section*{killall}
Finaliza todos los procesos que estan actualmente utilizando el/los procesos especificados.

\section*{ps}
Muestra informacion de una seleccion de procesos. Por defecto, ps selecciona todos los proceos con el mismo UID que el usuario actual y asociado con la misma terminal que el invocador.

\section*{pstree}
Muestra un arbol con los procesos corriendo en el sistema.

\section*{nice}
Ejecutar un programa con una prioridad de scheduling modificada.

\section*{tar}

Nos permite generar un solo archivo comprimido, que en su interior puede contener a un directorio completo o una serie de archivos.

\begin{itemize}

\item Para comprimir acrhivos:\\
\textbf{\path{tar -cvf dir_plano.tar /ruta/al/dir_plano/}}\\

\item Para mostrar el contenido dentro del paquete:\\
\textbf{\path{tar -tf dir_plano.tar}}

\item Para descomprimir el paquete:\\
\textbf{\path{tar -xvf dir_plano.tar}}\\
\end{itemize}

\end{document}

