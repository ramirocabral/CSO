\documentclass[11pt]{article}
\usepackage{enumerate}
%! TEX root = "~/home/ramiro/Desktop/CSO/practicas/practica1/main.tex"

\title{\Huge{Conceptos de Sistemas Operativos\\ }}
\author{\huge{Ramiro Cabral}}
\date{\today}

\begin{document}
\maketitle
\pagebreak
\section{Caracteristicas de GNU/Linux}

\subsection{Caracteristicas mas relevantes}
\begin{itemize}
    \item Es un sistema operativo UNIX-Like
    \item Es multiusuario
    \item Es multitarea
    \item Es altamente portable
    \item Es case-sensitive
    \item Es de codigo abierto
    \item Todo es un archivo(incluso los dispositivos y directorios)
\end{itemize}
\subsection{POSIX}
\begin{itemize}
    \item Conjunto de estandares que define varias interfaces de herramientas, comandos y API para Sistemas Operativos de tipo UNIX.
    \item Creada por la IEEE, define una interfaz estandar del sistema operativo y el entorno, inclutendo un interprete de comandos (shell).
\end{itemize}

\subsection{Distribuciones}
Una distribucion es una customizacion de GNU/Linux
formada por una version de kernel y determinados programas
con sus configuraciones.

\section{Estructura de GNU/Linux}
\subsection{3 componentes fundamentales de GNU/Linux}
\begin{itemize}
    \item Kernel(nucleo)
    \item Shell (interprete de comandos)
    \item FileSystem (sistema de archivos)
\end{itemize}

\section{Kernel de GNU/Linux}

\subsection{Que es el Kernel y cuales son sus funcines principales?}
Es el encargado de que el software y el hardware puedan trabajar juntos.
Sus funciones principales son:
\begin{itemize}
    \item Administracion de memoria.
    \item Administracion de la CPU.
    \item Administracion de la E/S.
\end{itemize}

\subsection{Arquitectura del Kernel}
El kernel es un nucleo monolitico hibrido:
\begin{itemize}
    \item Los drivers y el codigo del mismo se ejecutan en modo privilegiado.
    \item Lo que lo hace hibrido es la capacidad de cargar y descargar funcionalidad a traves de modulos.
\end{itemize}

\section{Shell (Interprete de comandos)}
\subsection{Que es?}
Es el modo de comunicacion que el usuario posee con el sistema operativo. El mismo ejecuta programas mediante el ingreso de comandos.

\section{File System (sistema de archivos)}
\subsection{Que es el Fyle System?}
Es una estructura de datos que utiliza el sistema operativo para controlar como se almacenan y recuperan los datos en un medio de almacenamiento.

\subsection{Jerarquia de directorios en GNU/Linux}
\begin{itemize}
    \item \textbf{/} : Tope de la estructura de directorios.
    \item \textbf{/home} : Se almacenan los archivos de los usuarios.
    \item \textbf{/var} : Informacion que varia de tamanio en el tiempo.
    \item \textbf{/etc} : Archivos de configuracion del sistema.
    \item \textbf{/bin} : Archivos binarios y ejecutables.
    \item \textbf{/dev} : Enlace a dispositivos.
    \item \textbf{/usr} : Aplicaciones de usuarios.
    \item \textbf{/tmp} : Archivos temporales.
    \item \textbf{/boot} : Informacion del booteo de la maquina.
\end{itemize}

\section{Bootstrap (Arranque del sistema)}
\subsection{BIOS(Basic I/O System)}
La BIOS es un chip instalado en la placa base con un firmware que contiene una serie de subrutinas basicas del procesador para el arranque del sistema. Actua como un intermediario entre la CPU y los dispositivos de I/O.

\subsection{MBR (Master Boot Record)}
\begin{itemize}
    \item Sector reservado del disco fisico (cilindro 0, cabeza 0, sector 1).
    \item Existe un MBR en todos los discos.
    \item Su tamano es de \textbf{512} bytes.
        \begin{itemize}
            \item 446 bytes corresponden al MBC (Master Bootloader Code).
            \item otros 64 bytes corresponden a la tabla de particiones del disco (partition table).
            \item Los ultimos 2 bytes son libres, pero pueden usarse para firmar el MBR.
            \item MBR tiene espacio acotado para la tabla de particiones, tenemos dos opciones:
                \begin{itemize}
                    \item 4 particiones primarias.
                    \item 3 primarias y una extendida con sus particiones logicas.
                \end{itemize}
        \end{itemize}
        \subsubsection{MBC}
        \begin{itemize}
            \item Es un pequeno codigo que permite arrancar el SO.
            \item La ultima accion del BIOS es ejecutar el MBC. Lo lleva a memoria y lo ejecuta.
            \item Si se tiene un SO instalado, generalmente corresponde al bootloader por defecto del sistema. Tambien puede instalarse en el un bootloader diferente (multietapa) como por ejemplo GRUB.
        \end{itemize}

        \subsection{GPT (Guid Partition Table)}
        El sistema GPT es un estandar de particiones de disco. Fue creado por Intel como reeemplazo para el estandar MBR.
        \begin{itemize}
            \item Utiliza un modo de direccionamiento logico (logical block addressing \textbf{LBA}) en lugar de la \textit{cylinder-header-sector} utilizada por MBR.
            \item Funciona con sistemas UEFI.
            \item Se mantiene un MBR legacy para mantener la compatibilidad con el esquema BIOS.
            \item LBA 0: MBR legacy.
            \item LBA 1: GPT Header.
            \item LBA 2...: Tabla de particiones.
            \item La cabecera GPT y la tabla de particiones estan escritas al principio y al final del disco por redundancia.
        \end{itemize}

        \subsection{Gestores de Arranque}
        Programas o software que se encargan de controlar el proceso de inicio de un sistema operativo. Su funcion principal es la de proporcionar al usuario la opcion de elegir que sistema operativo desea iniciar al prender la maquina.
        \begin{itemize}
            \item Carga una imagen del kernel de algun SO instalado en una particion.
            \item Se ejecuta luego del codigo de BIOS.
            \item Existen 2 modos de instalacion:
                \begin{itemize}
                    \item En el MBR (suele ubicarse en el MBC).
                    \item En el sector de arranque de la particion raiz o activa.
                \end{itemize}
        \end{itemize}
        \subsubsection{GRUB (Grand Unified Bootloader)}
        Gestor de arranque comun en sistemas Linux. 
        \begin{itemize}
            \item Altamente configurable, puede administrar multiples sistemas operativos y configuraciones del sistema.
            \item En el MBR se encuentra la fase 1, que luego carga la fase 1.5.
            \item La fase 1.5 se encuentra en los siguientes 30kb de disco. Carga la fase 2.
            \item La fase 2 posee la interfaz de usuario, y carga el Kernel seleccionado.
            \item Se configura a traves del archivo \textit{/boot/grub/menu.lst}.

                En la version 2 de GRUB, el archivo de configuracion se encuentra en \textit{/boot/grub/grub.cfg}
        \end{itemize}
\end{itemize}

\section{Proceso de arranque del sistema}
\begin{itemize}
    \item En las arquitectuas x86, el BIOS es el responsable de iniciar la carga del SO a traves del MBC
    \item Carga el programa de booteo (desde el MBR).
    \item El gestor de arranque lanzado desde el MBC carga el Kernel.
        \begin{itemize}
            \item Prueba y hace disponibles los dispositivos.
            \item Luego pasa el control al proceso \textbf{init}.
        \end{itemize}
\end{itemize}

\section{UEFI (Unified Extensible Firmware Interface)}
\begin{itemize}
    \item Define la ubicacion de gestor de arranque.
    \item Define la interfaz entre el gestor de arranque y el firmware.
    \item Expone infgormacion para los gestores de arranque con:
        \begin{itemize}
            \item Informacion de hardware y configuracion de firmware.
            \item Punteros a rutinas que implementan los servicios que el firmware ofrece a los bootloaders u otras aplicaciones.
            \item El bootloader ahora es un tipo de aplicacion UEFI. Ahora, para Grub, ya no es necesario el arranque en varias etapas.
        \end{itemize}
\end{itemize}
\subsection{Secure Boot}
\begin{itemize}
    \item Posee mecanismos para un arranque libre de codigo malicioso.
    \item Las aplicaciones y drivers UEFI son validadas para verificar que no fueron alteradas.
    \item Se utilizan pares de clabes asimetricas.
    \item Se almacenan en el firmware una serie de claves publicas que sirven para validar que las imagenes esten firmadas por un proveedor autorizado.
    \item Si la clave privada esta vencida o fue revocada la verificacion puede fallar.
\end{itemize}

\section{SystemV}
\subsection{Proceso de Inicio en GNU/Linux}
\begin{enumerate}
    \item Se comienza a ejecutar el codigo del Bios.
    \item El BIOS ejecuta el POST.
    \item El bios lee el sector de arranque (MBR).
    \item Se carga el gestor de arranque (MBC).
    \item El bootloader carga el kernel y el initrd.
    \item Se monta el initrd como sistema de archivos raiz y se inicializan componentes esenciales.
    \item El kernel ejecuta el proceso init y se desmonta el initrd.
    \item Se lee el /etc/inittab.
    \item Se ejecutan los scripts apuntados pr el runlevel 1.
    \item El final del runlevel 1 le indica que vaya al runlevel por defecto.
    \item Se ejecutan los scipts apuntados por el runlevel por defetco.
    \item El sistema esta listo para usarse.
\end{enumerate}

\subsection{Init}
\textbf{initrd} :Es cargado como parte del procedimiento de booteo del kernel. Contiene un set minimio de ejecutables y archivos necesarios para la inicializacion del kernel del sistema.
\begin{itemize}
    \item Carga los subprocesos necesarios para el correcto funcionamiento del SO.
    \item Posee el PID 1 y se encuentra usualmente en /sbin/init.
    \item En SystemV se lo configura a traves del archivo \textit{/etc/inittab}.
    \item No tiene padre y es el padre de todos los procesos (pstree).
    \item Es el encargado de montar los filesystems y de hacer disponible los demas dipositivos.
\end{itemize}

\subsection{RunLevels}
\begin{itemize}
    \item Hacen referencia al modo en que arranca Linux.
    \item Define que servicios del sistema estan operando.
    \item El proceso de arranque es dividido en diferentes niveles.
    \item Se encuentran definidos en \textit{/etc/inittab}.
    \item Syntaxis: \textbf{\textit{id:nivelesEjecucion:accion:proceso}}.
    \begin{itemize}
        \item \textbf{ID: }identifica la entrada en inittab.
        \item \textbf{nivelesEjecucion: }el/los niveles de ejecucion en los que se realiza la accion.
        \item \textbf{Accion: }describe la accion a realizar.
        \item \textbf{Proceso: }proceso exacto que sera ejecutado.
    \end{itemize}
    \item Existen 7 distintos RunLevels, y cada uno puede iniciar un conjunto de procesos al arranque o apagado del sistema.
    \item Los scripts que se ejecutan se encuentran en \textit{/etc/init.d}.
    \item En \textit{/etc/rcX.d} (con X entre 0 y 6) hay links a los archivos del \textit{/etc/init.d}.
    \item Formato de los links: \verb|[S-K]<orden><nombreScript>|.
    \item Para administrar el orden de los enlaces simbolicos, y resolver dependencias entre ellos se utiliza \textbf{insserv}.
    \item insserv usa cabeceras (De tipo LSB) en los scripts del init.d para especificar la relacion con otros scripts rc.
\end{itemize}

\section{Upstart}
\begin{itemize}
    \item Reemplazo propuesto para SystemV.
    \item Ejecucion de trabajos en forma asincronica a traves de eventos, a diferencia de SystemV que es estrictamente sincronico (\textit{dependency-based}).
    \item Estos trabajos se denominan \textbf{\textit{jobs}}.
    \item Los jobs definen servicios o tareas a ser ejecutadas por init.
    \item Son definidos en \textit{/etc/init}.
    \item Los jobs son ejecutados ante eventos, y es posible crear nuestros propios eventos.
    \item Cada job posee un objetivo (goal start/stop) y un estado (state).
\end{itemize}

\section{SystemD}

\begin{itemize}
    \item Sistema creado por RedHat (the devil himself) que centraliza la administracionde demons y librerias del sistema.
    \item Mejora el paralelismo de booteo.
    \item Puede ser controlado por el comando \textbf{\textit{systemctl}}.
    \item Compatible con SystemV.
    \item Los runlevels son reemplazados por \textbf{\textit{targets}}.
    \item Las unidades de trabajo son denominadas \textbf{\textit{units}} de tipo:
    \begin{itemize}
        \item \textbf{\textit{Service:}} controla un servicio particular.
        \item \textbf{\textit{Socket:}} encapsula, un IPC, un socket del sistema o filesystem.
        \item \textbf{\textit{Target:}} agrupa units o establece puntos de sincronizacion.
        \item \textbf{\textit{Snapshot:}} almacena el estado de un conjunto de unidades que puede ser restablecido mas tarde.
    \end{itemize}
\end{itemize}

\subsection{Activacion por Socket}
\begin{itemize}
    \item Es un mecanismo de iniciacion bajo demanda.
    \item Cuando el socket recibe una conexion spawnea el servicio y le pasa el socket.
    \item No hay necesidad de definir dependencias entre servicios, ya que se inician todos los sockets en primier medida.

\subsection{cgroups}
\begin{itemize}
    \item Permite organizar un grupo de procesos en forma jerarquica.
    \item Agrupa conjuntos de procesos realcionados.
\end{itemize}
\end{itemize}

\subsection{fstab}

\begin{itemize}
    \item Define que particiones se montan al arranque.
    \item Su configuracion se encuentra en \textbf{\textit{/etc/fstab}}.
\end{itemize}


\section{Usuarios en GNU/Linux}
\begin{itemize}
    \item Cada usuario debe poseer credenciales para acceder al sistema.
        \begin{itemize}
            \item \textit{root:} administrador del sistema (superuser).
            \item \textit{otros:} usuarios del sistema estandar (\textit{/etc/sudoers}).
        \end{itemize}
    \item Archivos de configuracion:
        \begin{itemize}
            \item \textbf{\textit{/etc/passwd}}
                \begin{itemize}
                    \item Cada linea representa a un usuario.
                    \item Posee informacion general del usuario (grupo,nombre,descripcion,homedir,shell,etc).
                \end{itemize}
            \item \textbf{\textit{/etc/group}}
                \begin{itemize}
                    \item Muestra cada grupo del sistema.
                    \item Cada usuario puede tener grupos secundarios.
                \end{itemize}
            \item \textbf{\textit{/etc/shadow}}
                \begin{itemize}
                    \item Posee nombres y contrasenas encriptadas (solo accesible por el root).
                \end{itemize}
        \end{itemize}
\end{itemize}

\subsection{UID y GID}

\begin{itemize}
    \item Un \textbf{UID} (User identifier) es un numero asignado por Linux a cada usuario del sistema. Este numero es usado para identificar el usuario y para determinar a que recursos puede acceder.
    \item Son almacenados en el archivo \textit{/etc/passwd}.
    \item El Root posee el UID 0.
    \item Las UID por debajo de 1000,por convencion, son reservadas para usuarios creados por el sistema, servicios y otras cuentas especiales.
    \item Al crear un nuevo usuario, se le asignara un UID mayor a 1000.
    \item Un \textbf{GID} (group identifier) es un numero asignado a cada grupo del sistema.
    \item Son almacenados en el archivo \textit{/etc/groups}.
    \item Los primeros 100 GIDs son reservados para el uso del sistema.
    \item El GID 0 corresponde al grupoo root.
    \item El GID 100 correponde al gruop 'users'.
\end{itemize}

\section{Procesos}
\begin{itemize}
\item Cuando un programa es ejecutado, el sistema le provee una instancia especial al proces. La instancia consiste en todos los servicios/recursos que podria utilizar el proceso en la ejecucion.
\item Un numero de ID de 5 digitos es asignado al proceso, este numero es denominado \textbf{PID}. Cada proceso posee un unico PID.
\item Un processo que se conecta a la terminal es llamado un \textbf{foreground job}. Es llamado asi ya que puede comunicarse con el usuario mediante la pantalla y el teclado.
\item Un proceso que se desconecta de la terminal y no puede comunicarse con el usuario es llamado un \textbf{background job}.
\item Si un proceso en background requiere inhteraccion con el usuario, el mismo se detiene y espera hasta establecer una conexion con la temrinal.
\end{itemize}

\section{Paquetes}
\begin{itemize}
    \item Los paquetes son archivos que contienen archivos, y no utilizan ningun tipo de compresion.
    \item Los archivos comprimidos representan la compresion, utilizando algun tipo de algoritmo, de uno o varios archivos comunes.
\end{itemize}

\section{Argumentos y valor de retorno}
\begin{enumerate}
    \item \textbf{\$0} contiene la invocacion del script.
    \item \textbf{\$1,\$2,\$3...} contienen cada uno de los argumentos.
    \item \textbf{\$\#} contiene la cantidad de argumentos recibidos.
    \item \textbf{\$*} contiene la lista de todos los argumentos.
    \item \textbf{\$?} contiene en todo momento el valor de retorno del ultimo comando ejecutado.
\end{enumerate}

\section{exit}
\begin{enumerate}
    \item Causa la terminacion de un script.
    \item Puede devolver cualquier valor entre 0 y 255.
    \begin{enumerate}
        \item El valor 0 indica que el script se ejecuto de forma exitosa.
        \item Un valor distinto indica un codigo de error.
        \item Se puede consultar el exit status imprimiendo la variable \$?.
    \end{enumerate}
\end{enumerate}

\end{document}
