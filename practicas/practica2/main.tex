\documentclass[11pt]{article}

\userpackage{url}
%! TEX root = "~/home/ramiro/Desktop/CSO/practicas/practica2/main.tex"

\title{\Huge{Conceptos de Sistemas Operativos\\
Practica 2}}
\author{\huge{Ramiro Cabral}}
\date{\today}

\begin{document}
\maketitle
\pagebreak

\section{SystemV}
\subsection{Proceso de Inicio en GNU/Linux}
\begin{enumerate}
    \item Se comienza a ejecutar el codigo del Bios.
    \item El BIOS ejecuta el POST.
    \item El bios lee el sector de arranque (MBR).
    \item Se carga el gestor de arranque (MBC).
    \item El bootloader carga el kernel y el initrd.
    \item Se monta el initrd como sistema de archivos raiz y se inicializan componentes esenciales.
    \item El kernel ejecuta el proceso init y se desmonta el initrd.
    \item Se lee el /etc/inittab.
    \item Se ejecutan los scripts apuntados pr el runlevel 1.
    \item El final del runlevel 1 le indica que vaya al runlevel por defecto.
    \item Se ejecutan los scipts apuntados por el runlevel por defetco.
    \item El sistema esta listo para usarse.
\end{enumerate}

\subsection{Init}
\textbf{initrd} :Es cargado como parte del procedimiento de booteo del kernel. Contiene un set minimio de ejecutables y archivos necesarios para la inicializacion del kernel del sistema.
\begin{itemize}
    \item Carga los subprocesos necesarios para el correcto funcionamiento del SO.
    \item Posee el PID 1 y se encuentra usualmente en /sbin/init.
    \item En SystemV se lo configura a traves del archivo \textit{/etc/inittab}.
    \item No tiene padre y es el padre de todos los procesos (pstree).
    \item Es el encargado de montar los filesystems y de hacer disponible los demas dipositivos.
\end{itemize}

\subsection{RunLevels}
\begin{itemize}
    \item Hacen referencia al modo en que arranca Linux.
    \item Define que servicios del sistema estan operando.
    \item El proceso de arranque es dividido en diferentes niveles.
    \item Se encuentran definidos en \textit{/etc/inittab}.
    \item Syntaxis: \textbf{\textit{id:nivelesEjecucion:accion:proceso}}.
    \begin{itemize}
        \item \textbf{ID: }identifica la entrada en inittab.
        \item \textbf{nivelesEjecucion: }el/los niveles de ejecucion en los que se realiza la accion.
        \item \textbf{Accion: }describe la accion a realizar.
        \item \textbf{Proceso: }proceso exacto que sera ejecutado.
    \end{itemize}
    \item Existen 7 distintos RunLevels, y cada uno puede iniciar un conjunto de procesos al arranque o apagado del sistema.
    \item Los scripts que se ejecutan se encuentran en \textit{/etc/init.d}.
    \item En \textit{/etc/rcX.d} (con X entre 0 y 6) hay links a los archivos del \textit{/etc/init.d}.
    \item Formato de los links: \verb|[S-K]<orden><nombreScript>|.
    \item Para administrar el orden de los enlaces simbolicos, y resolver dependencias entre ellos se utiliza \textbf{insserv}.
    \item insserv usa cabeceras (De tipo LSB) en los scripts del init.d para especificar la relacion con otros scripts rc.
\end{itemize}

\section{Upstart}
\begin{itemize}
    \item Reemplazo propuesto para SystemV.
    \item Ejecucion de trabajos en forma asincronica a traves de eventos, a diferencia de SystemV que es estrictamente sincronico (\textit{dependency-based}).
    \item Estos trabajos se denominan \textbf{\textit{jobs}}.
    \item Los jobs definen servicios o tareas a ser ejecutadas por init.
    \item Son definidos en \textit{/etc/init}.
    \item Los jobs son ejecutados ante eventos, y es posible crear nuestros propios eventos.
    \item Cada job posee un objetivo (goal start/stop) y un estado (state).
\end{itemize}


\section{SystemD}

\begin{itemize}
    \item Sistema creado por RedHat (the devil himself) que centraliza la administracionde demons y librerias del sistema.
    \item Mejora el paralelismo de booteo.
    \item Puede ser controlado por el comando \textbf{\textit{systemctl}}.
    \item Compatible con SystemV.
    \item Los runlevels son reemplazados por \textbf{\textit{targets}}.
    \item Las unidades de trabajo son denominadas \textbf{\textit{units}} de tipo:
    \begin{itemize}
        \item \textbf{\textit{Service:}} controla un servicio particular.
        \item \textbf{\textit{Socket:}} encapsula, un IPC, un socket del sistema o filesystem.
        \item \textbf{\textit{Target:}} agrupa units o establece puntos de sincronizacion.
        \item \textbf{\textit{Snapshot:}} almacena el estado de un conjunto de unidades que puede ser restablecido mas tarde.
    \end{itemize}
\end{itemize}

\subsection{Activacion por Socket}
\begin{itemize}
    \item Es un mecanismo de iniciacion bajo demanda.
    \item Cuando el socket recibe una conexion spawnea el servicio y le pasa el socket.
    \item No hay necesidad de definir dependencias entre servicios, ya que se inician todos los sockets en primier medida.

\subsection{cgroups}
\begin{itemize}
    \item Permite organizar un grupo de procesos en forma jerarquica.
    \item Agrupa conjuntos de procesos realcionados.
\end{itemize}
\end{itemize}

\subsection{fstab}

\begin{itemize}
    \item Define que particiones se montan al arranque.
    \item Su configuracion se encuentra en \textbf{\textit{/etc/fstab}}.
\end{itemize}

\section{Usuarios en GNU/Linux}
\begin{itemize}
    \item Cada usuario debe poseer credenciales para acceder al sistema.
        \begin{itemize}
            \item \textit{root:} administrador del sistema (superuser).
            \item \textit{otros:} usuarios del sistema estandar (\textit{/etc/sudoers}).
        \end{itemize}
    \item Archivos de configuracion:
        \begin{itemize}
            \item \textbf{\textit{/etc/passwd}}
                \begin{itemize}
                    \item Cada linea representa a un usuario.
                    \item Posee informacion general del usuario (grupo,nombre,descripcion,homedir,shell,etc).
                \end{itemize}
            \item \textbf{\textit{/etc/group}}
                \begin{itemize}
                    \item Muestra cada grupo del sistema.
                    \item Cada usuario puede tener grupos secundarios.
                \end{itemize}
            \item \textbf{\textit{/etc/shadow}}
                \begin{itemize}
                    \item Posee nombres y contrasenas encriptadas (solo accesible por el root).
                \end{itemize}
        \end{itemize}
\end{itemize}

\subsection{UID y GID}

\begin{itemize}
    \item Un \textbf{UID} (User identifier) es un numero asignado por Linux a cada usuario del sistema. Este numero es usado para identificar el usuario y para determinar a que recursos puede acceder.
    \item Son almacenados en el archivo \textit{/etc/passwd}.
    \item El Root posee el UID 0.
    \item Las UID por debajo de 1000,por convencion, son reservadas para usuarios creados por el sistema, servicios y otras cuentas especiales.
    \item Al crear un nuevo usuario, se le asignara un UID mayor a 1000.
    \item Un \textbf{GID} (group identifier) es un numero asignado a cada grupo del sistema.
    \item Son almacenados en el archivo \textit{/etc/groups}.
    \item Los primeros 100 GIDs son reservados para el uso del sistema.
    \item El GID 0 corresponde al grupoo root.
    \item El GID 100 correponde al gruop 'users'.
\end{itemize}

\section{Procesos}
\begin{itemize}
\item Cuando un programa es ejecutado, el sistema le provee una instancia especial al proces. La instancia consiste en todos los servicios/recursos que podria utilizar el proceso en la ejecucion.
\item Un numero de ID de 5 digitos es asignado al proceso, este numero es denominado \textbf{PID}. Cada proceso posee un unico PID.
\item Un processo que se conecta a la terminal es llamado un \textbf{foreground job}. Es llamado asi ya que puede comunicarse con el usuario mediante la pantalla y el teclado.
\item Un proceso que se desconecta de la terminal y no puede comunicarse con el usuario es llamado un \textbf{background job}.
\item Si un proceso en background requiere inhteraccion con el usuario, el mismo se detiene y espera hasta establecer una conexion con la temrinal.
\end{itemize}

\section{Paquetes}
\item Los paquetes son archivos que contienen archivos, y no utilizan ningun tipo de compresion.
\item Los archivos comprimidos representan la compresion, utilizando algun tipo de algoritmo, de uno o varios archivos comunes.
\item 

\end{document}
